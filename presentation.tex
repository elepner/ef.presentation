
\documentclass{bredelebeamer}

\usepackage[T1]{fontenc}
\usepackage[scaled]{beramono}

\usepackage{color}
\definecolor{bluekeywords}{rgb}{0.13,0.13,1}
\definecolor{greencomments}{rgb}{0,0.5,0}
\definecolor{redstrings}{rgb}{0.9,0,0}

\usepackage{listings}
\lstset{language=[Sharp]C,
showspaces=false,
showtabs=false,
breaklines=true,
showstringspaces=false,
breakatwhitespace=true,
escapeinside={(*@}{@*)},
commentstyle=\color{greencomments},
keywordstyle=\color{bluekeywords}\bfseries,
stringstyle=\color{redstrings},
basicstyle=\ttfamily
}

\title[EFC Tips \& Tricks]{Entity Framework Core: Tips and Tricks}

% \subtitle{Some subtitle}

\author{Eduard Lepner\inst{1}}
\institute[Powel]
{
  \inst{1}%
  Powel AS\\
  Team Leader at Powel Water
  }


\date{April 2018}

\subject{EF Core: Tips and Tricks}
% Goes to PDF Metadata

\logo{
\includegraphics[scale=0.10]{images/logo.jpg}
}

\begin{document}

\begin{frame}
  \titlepage
\end{frame}





% \begin{frame}{Table of contents}
%   \tableofcontents
% \end{frame}




\section{Level Easy}



\begin{frame} {About Entity Framework}
    \begin{columns}[t]
        \begin{column}{0.5\textwidth}
            Queries\\[.2cm]
            \begin{enumerate}
                \item Queries (Where clauses, Take, Skip, OrderBy)
                \item Projections (Select statement)
                \item Aggregate (Min, Max, GroupBy)
            \end{enumerate}
            \onslide<2->{
                \begin{itemize}
                    \item Expression analysis
                    \item Expression translations
                    \item SQL Builder
                \end{itemize}
            }
            
        \end{column}
        \begin{column}{0.5\textwidth}
            Commands\\[.2cm]
            \begin{enumerate}
                \item Create
                \item Read
                \item Update
                \item Delete
            \end{enumerate}
            \onslide<3->{
                    \begin{itemize}
                        \item Dynamic Proxies
                        \item Changes Detection (Object graph tracking)
                    \end{itemize}
                }
        \end{column}
    \end{columns}
\end{frame}

\begin{frame}{Expression Trees}
    \begin{block}{Any valid OO, Procedural, Declarative or Functional code}
        can be translated to an Abstract Syntax Tree (AST)...
        \pause  ... aaaand vice versa
    \end{block}
    \pause
    \begin{block}{AST}
        is a language indepent abstraction \pause (that's why FP is cool)
    \end{block}
    \pause
    \begin{block}{Expression Tree}
        is an AST. Expression yields one value and AST may represent entire program
    \end{block}
    
    \begin{alertblock}{Bingo!}
        Language A $\Rightarrow$ AST $\Rightarrow$ Language B
    \end{alertblock}
\end{frame}

\begin{frame}[containsverbatim]{Example}
    \begin{lstlisting}
/**
* Prints Hello World.
**/
class Program
{
    public static void Main()
    {
        System.Console.WriteLine("Hello World!");
    }
}
    \end{lstlisting}
\end{frame}




\section{Les blocs}

\begin{frame}{Les blocs}

\begin{block}{Bloc simple}
\begin{itemize}
\item Premier point
\item Second point
\end{itemize}
\end{block}

\begin{exampleblock}{Bloc exemple}
\begin{itemize}
\item Premier point
\item Second point
\end{itemize}
\end{exampleblock}

\begin{alertblock}{Bloc alert}
\begin{itemize}
\item Premier point
\item Second point
\end{itemize}
\end{alertblock}
\end{frame}


\section{Les bo\^ites}

\begin{frame}{Les boites}

\begin{columns}

\begin{column}{0.5\textwidth}
\boitejaune{
Ceci est \\
une boite jaune
}

\boiteorange{
Ceci est \\
une boite orange
}

\boitemarron{
Ceci est \\
une boite marron
}
\end{column}

\begin{column}{0.5\textwidth}
\boiteviolette{
Ceci est \\
une boite violette
}

\boitebleue{
Ceci est \\
une boite bleue
}

\boitegrise{
Ceci est \\
une boite grise
}

\end{column}

\end{columns}


\end{frame}




\section{Le texte}

\begin{frame}{Titre de la frame} 

Voici du texte normal

\alert{Voici du texte \texttt{alert}}

\exemple{Voici du texte \texttt{exemple}}

\emph{Voici du texte \texttt{emphase}}

\end{frame}


\section{Les tableaux}

\begin{frame}{Tableaux}

% merci: http://tex.stackexchange.com/questions/112343/beautiful-table-samples

\begin{tcolorbox}[tabjaune,tabularx={X||Y|Y|Y|Y||Y}, boxrule=0.5pt]
Couleur & Prix 1  & Prix 2  & Prix 3   & Prix 4   & Prix 5 \\\hline\hline
Rouge   & 10.00   & 20.00   &  30.00   &  40.00   & 100.00 \\\hline
Vert    & 20.00   & 30.00   &  40.00   &  50.00   & 140.00 \\\hline
Bleu    & 30.00   & 40.00   &  50.00   &  60.00   & 180.00 \\\hline\hline
Orange  & 60.00   & 90.00   & 120.00   & 150.00   & 420.00
\end{tcolorbox}

\begin{tcolorbox}[tabvert,tabularx={X||Y|Y|Y|Y||Y}, boxrule=0.5pt, title=Mon tableau des prix]
Couleur & Prix 1  & Prix 2  & Prix 3   & Prix 4   & Prix 5 \\\hline\hline
Rouge   & 10.00   & 20.00   &  30.00   &  40.00   & 100.00 \\\hline
Vert    & 20.00   & 30.00   &  40.00   &  50.00   & 140.00 \\\hline
Bleu    & 30.00   & 40.00   &  50.00   &  60.00   & 180.00 \\\hline\hline
Orange  & 60.00   & 90.00   & 120.00   & 150.00   & 420.00
\end{tcolorbox}

\end{frame}


\begin{frame}{Tableaux}

% merci: http://tex.stackexchange.com/questions/112343/beautiful-table-samples

\begin{tcolorbox}[tabgris,tabularx={X||Y|Y|Y|Y||Y}, boxrule=0.5pt]
Couleur & Prix 1  & Prix 2  & Prix 3   & Prix 4   & Prix 5 \\\hline\hline
Rouge   & 10.00   & 20.00   &  30.00   &  40.00   & 100.00 \\\hline
Vert    & 20.00   & 30.00   &  40.00   &  50.00   & 140.00 \\\hline
Bleu    & 30.00   & 40.00   &  50.00   &  60.00   & 180.00 \\\hline\hline
Orange  & 60.00   & 90.00   & 120.00   & 150.00   & 420.00
\end{tcolorbox}

\begin{tcolorbox}[taborange,tabularx={X||Y|Y|Y|Y||Y}, boxrule=0.5pt, title=Mon tableau des prix]
Couleur & Prix 1  & Prix 2  & Prix 3   & Prix 4   & Prix 5 \\\hline\hline
Rouge   & 10.00   & 20.00   &  30.00   &  40.00   & 100.00 \\\hline
Vert    & 20.00   & 30.00   &  40.00   &  50.00   & 140.00 \\\hline
Bleu    & 30.00   & 40.00   &  50.00   &  60.00   & 180.00 \\\hline\hline
Orange  & 60.00   & 90.00   & 120.00   & 150.00   & 420.00
\end{tcolorbox}

\end{frame}



\section{Les images}

\begin{frame}{Titre de la frame} 

\begin{figure}
\centering

% \caption{Éléments d'architecture bretonne typique du Sud de la France. (\href{http://commons.wikimedia.org/wiki/File:Colmar_-_Alsace.jpg}{Wikipédia.fr} CC-By-Sa)}
\end{figure}

\end{frame}



\end{document}
